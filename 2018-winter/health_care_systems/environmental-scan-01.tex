\documentclass[11.5pt]{article}
\usepackage{hyperref}
\begin{document}

\title{Environmental Scan \#1}
\author{Zelly Snyder}
\date{\today}
\maketitle

\section{Article: ``Physicians concerned about new opioid pharmacy rules''}
Cole, LaDyrian. ``Physicians concerned about new opioid pharmacy rules''. CBS [Internet]. Published 2018 Dec 11 [cited 2018 Dec 12]. Available from: \href{https://www.cbs19.tv/article/news/local/physicians-concerned-about-new-opioid-pharmacy-rules/501-623011135}{https://bit.ly/2Eg2zed5}

\section{Summary}
The Texas State Board of Pharmacy recently created several new pharmacy prescription regulations that give more responsibility to pharmacists to judge the legitimacy of prescriptions before agreeing to dispense them. For example, if a patient officially resides hundreds of miles away, a pharmacist may interpret this as a red flag. Under the new guidelines, it is appropriate for the pharmacist to call the prescriber and ask for clarification about the diagnosis or indication for which the patient is receiving an opioid prescription. These guidelines were implemented in response to the President's declaration of the opioid crisis as a Public Health Emergency last year. Physicians in Texas are responding to these changes with the concern that these new changes are straining the traditional physician-patient-pharmacist relationship.

\section{Reflection}
I see these new changes as beneficial for both public health and the pharmacy industry.

The opioid crisis is obviously a serious public health problem. Lax healthcare standards are generally believed to contribute to this problem. Before a patient gets an opioid prescription, there are several healthcare workers with whom they come in contact; anywhere along that path, they may be denied access to the drug. Pharmacists serve as the final gatekeeper before a patient gets an opioid medication in their hands---usually believed to be the on-ramp to most opioid addictions. It makes sense to strengthen this checkpoint first.

This isn't a transfer of responsibility for medical practice from physicians to pharmacists. It is an addition to the pharmacist's responsibility without diminishing the physician's responsibility at all. The physician still can perform due diligence before writing an opioid prescriptions. (In fact, this new rule encourages due diligence so as to avoid a veto by the pharmacist.) I argue that no one loses, and the public wins an extra safeguard against new or continuing opioid addictions. The healthcare industry ought to use every method at its disposal to fight against public health crises instead of fueling them.
\end{document}
